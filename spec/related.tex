\section{Cloud Storage Synchronization Literature Search}

Provenance
	JMB
	Nov. 2011
	Scope: 
		- Interested in determining current solutions for storing sensor data in the Cloud
		- required understanding of:
			* general cloud computing ideas and terminology
			* specific connections between Cloud Computing and Sensors
			* cloud (and general file/data) synchronization options
			* database replication
			* Weak n/w connection handling
		- During search expanded notion from simple synchronization at source to cloud gateway

	Methods:
			Google scholar
			Wikipedia
			Paper references
			alternativeto
			ieee
			acm

Outcomes			
	
	Cloud Computing And Sensor Data
		Observations
			- Most resources are orientated at the Cloud-level (that is to tsay the back-end data centres) rather than the data input level
			- ``Capacities continue to double each year, while access times are improving at 10 percent per year.'' - Jim Grey http://queue.acm.org/detail.cfm?id=864078
			- The Big 4 (Amazon, Microsoft, Google and Rackspace) offer various storage models
			- According to http://www.broadbandspeedchecker.co.uk/: 31.21 Mb/s download (337068 MB/d ), 13.29 Mb/s upload (143532 MB/d) here at Horizon
		
		General resources
			Above the Clouds: A Berkeley View of Cloud Computing
				- Definitions: 	* cloud = Datacentre h/w + s/w
								* SaaS = Services run on top of datacentre
				- Opportunities:	* The cloud will become dominant because it is the most convenient way to store automated (large) datasets from sensors and users -- services can then process, analyse and mashup the highly available data 
									* infinite capacity on-demand
				- Obstacles:	* Service availability
								* Data lock-in -- possibly mitigated by API standardization
								* Data confidentiality -- mitigated by encryption prior to storage
								* Data transfer bottlenecks -- costs (\$100-\$150 / TB -- unclear where figure comes from; )
																- SneakerNet is probably not viable here [Automated SneakerNet for smart buildings???]
			
			Cloud Storage: Adoption, Practice and Deployment
				- Definitions: 	* cloud computing = WAN infrastructure
								* Cloud (with a capital C) = the public data network including the Internet
				- Provides a good overview of various issues such as security and data verification, performance, inter-cloud portability
				- Block Systems, File Systems (CIFS, NFS and GFS) and Object Storage


			http://gigaom.com/2011/11/29/when-can-we-stop-talking-about-the-cloud/
				``If I was explaining either service to my parents, for example, I would try not to use the term ``cloud'' at all. It wouldn''t really make any sense to them without an explanation, and once I started explaining it -- how the cloud is a bunch of servers that Amazon or Google maintains in giant buildings that hold billions of individual files -- it would actually make things worse. If I just pointed out that any files placed in a specific folder would automatically show up in other folders on different computers, then they would understand everything they need to know.''
			
			Cloud growth:
				http://gigaom.com/cloud/amazon-storing-more-than-449-billion-objects-in-s3/
				http://gigaom.com/cloud/rackspace-cloud-revenue-keeps-rising-tops-100m/
			
			Basic principles of Machine-to-Machine communication and its impact on telecommunications industry
				http://www.ericsson.com/res/docs/whitepapers/wp-50-billions.pdf
	
		Sensor-Cloud
			Can We Plug Wireless Sensor Network to Cloud?
				- Raw data vs events
					simple or complex events
				- Proposed platform architecture (unclear where platform components live)
					[Of interest to this work (other components are downstream to sync data)]
					* WSN virtualisation manager
						- adapter provides raw byte stream from WSN to gateway which gateway enques as a raw packet in a buffer
						- Data processors affect enqueued packets
						- Command interpreter reprograms sensors via the gateway
					* Computation and storage manager
						- Processes and archives sensor data as XML
			% System Architecture Directions for Tangible Cloud Computing
				% - Excluded on the basis of not solving sync problem
			A framework of sensor-cloud integration opportunities and challenges
				- Goal: connect sensors, people and s/w to build community-centric sensing applications
				- "current Cloud providers did not address the issue of integrating sensor network with Cloud applications and thus have no infrastructure to support this scenario"
				- Cloud vs grid (and how cloud is more appropriate for WSN data, but wsn-cloud integration is an unaddressed issue )
				- Proposed solution
					- Sensors stream data to gateway
					- Gateway sends data to pub-sub system in the cloud
					- Cloud services provide event monitoring and application services
			A design of flexible data channels for sensor-cloud integration
				- Attempts to integrate data channels from sensors to data analysis jobs in cloud
				- Their tools include Sensor Server
					- Runs on SN master node
					- activate/deactivate slave nodes
					- filter sn data
					- deliver data to clouds
					- Acts as a pull-based FTP server that makes sensors accessible to cloud services as files
				- Comms models
					- Push based
						* 1 data point = UDP packet or SNMP packet
					- Pull based				
			WhereStore: location-based data storage for mobile devices interacting with the cloud
				%- [A lot of the paper would be useful when considering predicitve storage patterns, however notes about their location prediction have been excluded from this lit revie]
					% - "Predicting an application's data pattern is one problem, deciding the exact time at which data should be fetched from the cloud is yet another." 
					% - "a system like WhereStore can use information about current and future Internet access opportunities as well as information about different access technologies (e.g. 3G or WiFi) to make decisions about when to communicate with a remote cloud storage service. An intelligent decision may reduce the total power consumption of the device without data access penalties"
				- Filtered replication
					- keep data synced b/w set of peers
						- collections of items (each item containing data and metadata)
						- replicas of entire collections or subsets
						- filters (predicates over metadata) describe subsets
				- architecture
					- phone client, cloud server
					- WhereStore is stacked on top of:
						-filtered replication system
							- sync local cache on phone with cloud data items
							- smartphone replica contains items defined by its filter
								- subset of items for which totla capacity is below specified storage capacity
								- selection algorithm is given in paper
							- Based on Cimbiosys
								- Data accessed through callbacks
						-location service
						
			A survey of mobile cloud computing
				- mobile cloud computing (MCC)
				- obstacles / MCC advantages
					performance 
						battery life / Extending battery lifetime
						storage / Improving data storage capacity
						low bandwidth
					environment
						heterogeneity
						scalability / MCC apps can be scaled to meet the unpredictable user demands
						availability
					security
						reliability / Improving reliability (reduces the chance of data and application lost)
						privacy
							data integrity / [92] presents energy efficient solution
							authentication / [93] presents method for secure the data access suitable for mobile environments
					data access
							mechanism to manage, and synchronize data between mobile devices and the cloud to deal with changes
of data on the cloud
						[102] addresses three main issues as follows: maintaining seamless connection between users and
clouds, controlling cache consistency, and supporting data privacy

				- Mobile Cloud Computing (MCC) architecture(Fig. 1)
				- Cloudlets: trusted, resource-rich computer or cluster of computers which is wellconnected
to the Internet and available for use by nearby mobile devices
			Usable, lightweight and secure, architecture and programming interface for integration of Wireless Sensor Network to the Cloud.
				*** The following notes are direct copies of text:
				- The integration of wireless sensor networks, aggregated with sensor data but with limited processing power, with a resource-rich cloud computing infrastructure makes the integration beneficial in terms of i) scalability of data storage, ii) scalability of processing power for data mining and analysis, iii) access to the processing and storage infrastructure anywhere in the world, iv) resource optimization, v) and using pricing as one more criteria for the IT infrastructure.
				- Intortus, to enable this exploration of integrating sensor networks to the emerging cloud computing paradigm.
				- Intortus wherein sensor nodes collect data and send data to a gateway. The gateway aggregates data from the Wireless Sensor Network (WSN).
				- gateway in Intortus uses the same hardware as the sensor node (a wall powered mote)
				- web API with defined methods for accessing data
				- Intortus sends data from the gateway to the cloud securely using message authentication and encryption
				- user authentication data access policy and gateway authentication
				- Intortus currently, deals with a single application and is a framework dealing with solely data storage and communication to and fro the cloud in a secure manner
				- Intortus system consists of a cloud service that will be used to add data from the gateway to the cloud data store
				- Intortus provides the above first two services:
					1. Store data from the WSNs securely.
					2. Make this data available to users for viewing through web pages, to smart phone applications which will connect to Intortus and to other programs which will connect directly to Intortus
					3. Get input from users via web pages, smart phone apps or other programs to control the applications in the WSNs.
					4. Allow applications to be loaded from desktop computers through the cloud service to the WSN gateways, which will then push them onto individual motes in the WSN.
				- eMotes are loaded with software to collect, encode, and transmit data through wireless communication channels to the gateway
				- gateway is responsible to receive data from sensors and to dispatch it to be appropriate storage service hosted on cloud. The gateway structures data in the form of an XML and sends it over to the cloud. The data is sent to the cloud using web API method calls. The web API is created from a web service which is hosted in the cloud for interoperability from .NET Micro framework at the gateway to Java in the cloud.
				- data storage is Google App Engine
				- Intortus data storage API, used to store data from data aggregated at the gateway, creates a relational interface to the key value data structure so that the application providers need no knowledge of the Google App Engine data store structure while inserting data to the cloud. Data in the cloud is stored in an encrypted form for security purposes. Data can be accessed by an API which is a REST
				- Security in Intortus includes message authentication and encryption, data access, user authentication and gateway authentication
				- Challenges:
					* limited sensor node hardware resources make it hard to implement a security platform that incorporates the security requirements of Intortus.
			An autonomic cloud environment for hosting ECG data analysis services
				- Of little interest
			Cloud4Home-- Enhancing Data Services with AtHome Clouds
				*** The following notes are direct copies of text (unless prefaced by ***):
				- current 'thin client' models in which end devices 'simply access the Internet' can suffer from high and variable delays in accessing and using remote resources
				- they are subject to challenges when devices must operate in disconnected
				- should leverage the lower costs of using local resources and exploiting locally available state, avoid potential issues with data privacy or security for cloud-based operation
				- should exploit Internet resources when those are not encumbered by undue costs like high latency or undue communication overheads
				- data services explored in the paper can tap into the aggregate resources offered by remote clouds, and they can leverage 'nearby' devices in home or office settings. The outcome is quality in service delivery that exceeds that of the pure 'in the cloud' or 'at the edge' service realizations.
				*** not really cloud is it -- data centre oriented...
				- focus n services for storage, access, and manipulation of data for the home environment and when doing so, we leverage the VStore++
					- virtualized object storage system that abstracts from an application where the objects it accesses are stored
					- VStore++ will track resource availability in order to direct requests to appropriate destinations
based on their needs and/or resource availability, using a global indexing and monitoring infrastructure maintained
during its operation
					- Cloud4Home comprised of dynamically varying sets of devices that cooperate to provide end users with seamless storage, access, and data manipulation services, including interactions with remote,
publically available cloud platforms
					- vStore++ uses a standard file system to represent objects, using a one-to-one mapping
of objects to files. In addition to object fetch and store operations, it supports an explicit process operation, which
permits object manipulation functions to be associated with the object access								*** overkill for WSN???
					- metadata layer provides object lookup and transparent access to storage and services distributed across nodes in the home or remote clouds
					 - metadata management layer is built as a distributed key-value store on top of a peer-to-peer overlay across all control domains in the home cloud
					 - One or more nodes in the home cloud support a public cloud interface module, responsible for routing all remote cloud interactions
			Towards Reliable, Performant Workflows for Streaming-Applications on Cloud Platforms
			*** The following notes are direct copies of text (unless prefaced by ***):
				- propose and present a scientific workflow framework that supports streams as first-class data, and is optimized for performant and reliable execution across desktop and Cloud platforms
				- combine streaming data arriving from sensors with historic data available in file archives along with structured collections of weather forecast data that help the large scale computational model make an energy use prediction in near real time
				- workflow architecture that natively supports the three common data models found in science and engineering applications -- files, structured collections and data streams -- with the ability to seamlessly transition from one data model to another
				- logical stream abstractions have to be more robust than simple TCP sockets, given the unreliability and opaqueness introduced by operating in a distributed environment across desktop and Cloud with different characteristics from a typical local area network. Reliability of VMs hosting workflow tasks
is another concern to be addressed. Too, there has to be intelligence to avoid costly (both in time and money) duplicate movement of the same logical stream
				- Streams are a continuous series of binary data
				- transient unless mapped to another data model
				- In the future, Cloud providers may provide such streams as IaaS abstraction
				- registry service that maintains a list of known streams and the endpoints where particular
streams are provided
				- fault resistance: (1) transient or permanent loss of physical network, and (2) loss of virtual machines in the Cloud or services running on them.
				
	Cloud experiences
			Adaptive Rate Stream Processing for Smart Grid Applications on Clouds
				- adaptive rate control to throttle the rate of generation of power events by smart meters, which meets accuracy requirements of smart grid applications while consuming 50\% lesser bandwidth resources in the Cloud
				- power events generated at a peak rate of 1 KB/min from 1.4 M consumers in the Los Angeles Smart Grid will require 2 TB/day of streaming data to be processed and analyzed at an average cumulative bandwidth of ~200Mbps
				- need for an adaptive stream rate control mechanism for generating power usage events
				- use the difference between the available power capacity at the utility and the current cumulative power usage by the consumers as our throttle function
				- disadvantages of static publishing rate:
					- setting too high a rate at which power events are published will cause excess resources (bandwidth, compute VMs for stream processing) to be utilized
					- setting too low a static rate at which power events are published can cause the utility to miss detecting a breach of a power usage threshold during peak load periods
				- adaptation logic (throttle)
					- as the total power usage within the utility approaches total available capacity, power usage events are required more frequently to detect/forecast a peak load event with low latency.
				- As future work, we plan to evaluate the scalability of the stream processing system with increasing number of VM instances, with the eventual goal of scaling up to 1.4 million smart meter streams that is expected in the Los Angeles Smart Grid. Both throughput and latency of the pipeline need to be measured as the stream rates adapt. Additional factors like availability of compute resources (\# of VMs), throughput of the stream processing system, and cost trade-off between Cloud resource usage and power conserved can be incorporated into the throttle policy.		
			
	Cloud Storage Gateway
		General resources
			wikipedia.org
				A cloud storage gateway is a network appliance or server which resides at the customer premises and translates cloud storage APIs such as SOAP or REST to block-based storage protocols such as iSCSI or Fibre Channel or file-based interfaces such as NFS or CIFS.
			
			CloudStorageSurvey--Avere
				- make cloud storage appear as local storage
				- translates cloud storage APIs to file interfaces or storage transport protocols
				- gateway resides at customer site
					- can be dedicated appliance/server, virtual appliance/server, or s/w application
						- features: compression, deduplication, encryption, caching, backup and recovery
				- Provides useful table columns for comparing cloud grteways, but unfortunately the table is incomplete (only the Avere FXT was examined and these cost \$50000+)		
			
			http://gladinet.blogspot.com/2010/09/how-to-pick-cloud-storage-gateway.html 
				- pick a cloud storage gateway, there are several questions:
					- 3 forms: s/w, virt. applicance, physical
					- which supported cloud storage providers
					- how to store data on cloud: 
						- block level
						- file level
					- what encryption/compression algo used
					
			http://www.cloudswitch.com/page/now-everybody-wants-a-cloud-gateway
				- gateway simplifies the integration and management of cloud resources so people can get on with using the cloud rather than struggling to make it work
				
			http://www.cloudswitch.com/blog/category/Cloud%20Gateway%20Series
				- a well-designed cloud gateway needs to:
					- guarantee security
					- gateway should allow users and administrators to monitor and manage applications running in a cloud as if they were running locally, using existing tools and polices in a single, integrated environment
					- protect roles and access
					
www.nasuni.com/blog/28dirtysecrets5weaknessesofcloudstorage  
				- gateway or device (virtual or physical) that needs to be installed at your site
				- beenfits:
					- improved performance
					- security
					- compression
					- dedup
					- WAN optimization
					- cache data
				- limitations
					- internet connection (can be mitigated usinga bulk data migration service)
					- Caches are not perfect and sometimes you will get read and write misses 
						- write miss: the space you need to write to in the cache is temporarily full and has not been freed up by offloading/sending data to the cloud
							-  ability to feed data to the appliance most likely is significantly faster than your Internet connection 
					- You can't get at your data without the appliance because the appliance adds a level of security
				- cloud storage gateways change the data written to them
					- compression, dedeup, encrypted, chunked for parallelism
					
			http://www.nasuni.com/blog/32blocksvsfileswhichapproachisbetterforcloud
				- block-based gateway
					- gateway works directly with blocks, the 512-byte fundamental units of storage
					- raw chunks of storage on a drive
					- Applications assume that blocks exist in local, fast storage
						- cloud: connections between the applications and the storage can become strained (latency issues)
					- Intelligent caching is really, really difficult 
					- Block-based gateways restore at the volume level, essentially restoring the blocks in a volume to a previous point in time
				- file-based gateway
					- Files provide a clear picture of the relationships among the blocks that support them
					- File-based gateways provide restore capabilities at the file, directory, or complete file system level
					- best for unstructured data
		Commercial offerings
			Ctera
				- cloud storage gateways that provides shared storage, cloud backup, folder synchronization and remote access in single devices
					- only uploads diffs/changed data
					- de-duplication
					- user bandwidth throttling (time based or constant)
					- 128-bit SSL
					- data encrypted using 256-bit AES & fingerprinted by 160 bit SHA-1
					- Restore all or individual files
					- Rules-based CIFS, WebDAV and rsync based sync
					- FTP
					- Dashboard
						- Resource utilization
						- Event logs (system, access, backup, sync, audit)
						- Backup schedules
					- Email alerts
					- Automatic updates
					- Config backup/restore
				- rack mounted and plug form factors
				- remote centralised monitoring/management through a portal (SaaS or installed software)
			Nasuni
				- Nasuni Filer, a storage controller that runs in a datacenter as a hardware appliance or a virtual appliance
					- primary storage
					- built-in backup
					- offsite backup
					- all data and metadata is stored in the cache
					- cache snapshots
						- each file is encrypted, compressed and backedup to the Nasuni Service
					- cache management removes rarely re-used docs to respect the cache capacity
				- portal provides
					- volume & cache metrics
					- alerts/notifications
					- remote file accss
					- control file sharing in the network
					- restore from snapshot
			Citrix
			http://www.opensourcerack.com/2011/05/25/citrix-netscaler-cloud-gateway-a-product-tour/
				- gateway to provision application access to users by administrators
			Gladinet 
				- Windows clients
				- Cloud server (http://gladinet.blogspot.com/2011/10/introducing-gladinet-cloud-server.html)
					- on premise gateway
					- file server that allows you to mount different cloud storage services such as Amazon S3 (default), Windows Azure, Google Storage, EMC Atmos into a Windows File Server
					- Map Drive to the Attached Cloud Storage Folder
					- Cache 
					- Connected to account
					- Mount service accounts (such as Amazon S3)
					- Active directory integration
				- Cloud for teams service
			Atmos 
				http://www.emc.com/collateral/software/white-papers/h9505-emc-atmos-archit-wp.pdf
				- store, manage, and protect globally distributed, unstructured content at scale
				Intel cloud builders guide
				http://www.emc.com/collateral/software/data-sheet/h5770-atmos-ds.pdf
		Open
			http://openstack.org/
				- backend oriented (S3 competitor)
			Nimbus Project
				https://github.com/nimbusproject/nimbus
				http://www.nimbusproject.org	
					- cloud computing for science
					- allows a client to lease remote resources by deploying virtual machines (VMs) on those resources and configuring them to represent an environment desired by the user					
				Cumulus
					Cumulus: an open source storage cloud for science
						- storage cloud implementation
						- compatible with the Amazon Web Services S3 REST API
						- quota management
						- used against many existing clients (boto, s3cmd, jets3t, etc)
						- open source implementation of the Amazon S3 REST AP
						- configure Cumulus with existing systems such as GPFS, PVFS and HDFS, in order to provide the desired reliability, availability or performance trade-offs
						- upload data to the cloud, monitor its status, and download it from the storage cloud as needed.
						- provides an image store for Nimbus compute clouds
						- S3 interface allows clients to write, read, and delete objects or organize them into buckets
						- Authentication mechanisms, based on request signature by symmetric key
						- authorization database
						- Redirection module used to handle scalability
						- implemented in python
						- the concept of a storage cloud is a fusion between data transfer and storage management
			Eucalyptus
				http://www.cca08.org/papers/Paper32-Daniel-Nurmi.pdf		
					- opensource software framework for cloud computing that implements IaaS
			
			JetS3t
				http://jets3t.s3.amazonaws.com/index.html
				- free, open-source Java toolkit and application suite for Amazon Simple Storage Service (Amazon S3), Amazon CloudFront content delivery network, and Google Storage for Developers
				- 5 applications:
					- Cockpit: graphical application for transferring files, viewing and managing the contents of an Amazon S3 or Google Storage account
					- Synchronize: command-line application for synchronizing directories on your computer with an Amazon S3 or Google Storage account
						- Files are copied to/from the Amazon S3 or Google Storage service
						- by default, only new or changed files are transferred
						- synchronize.properties file
							- accesskey and secretkey which define AWS access credentials
							- upload.max-part-size
								- files larger than this value will be split into smaller parts no larger than the value and uploaded as Multipart Uploads
							- upload.ignoreMissingPaths
								- Synchronize will perform an upload despite missing or unreadable source files
							- files can be gzipped or encrypted during synchronization
							- upload.metadata
								-  Custom metadata to apply when uploading new files to S3
						- no action option to generate a report of what will happen
						- force sync when files are up-to-date
					- Gatekeeper: servlet that acts as an authorization service running on a Service Provider's server to mediate access to S3 accounts
					- CockpitLite: graphical application that Service Providers with S3 accounts may provide to clients or customers without S3 accounts
					- Uploader: graphical application that Service Providers with S3 accounts may provide to clients or customers without S3 accounts
					
		s3cmd
			http://s3tools.org/s3cmd
				- command line tool for uploading, retrieving and managing data in Amazon S3
				- conditional/unconditional transfer
				- http and https
				- GPG encryption
				
	Synchronization Tools
		General resources
			Synch comparison chart 
				en.wikipedia.org/wiki/Comparisonoffilesynchronizationsoftware
		
		Common synch tools
			rsync
				- invented by Andrew Tridgell
				- first announced on 19 June 1996.[1] Rsync 3.0 was released on 1 March 2008
				- requires an rsync server running rsync daemon
				- algorithm
					- synchronizes files and directories from one location to another while minimizing data transfer using delta encoding
					
				- application
					- standard Linux utility
					- ported to Windows (via Cygwin), Mac OS
					- rsync is capable of limiting the bandwidth consumed during a transfer
					- supports compression and decompression
					- SSH
			DeltaCopy
				- open source Windows' wrapper of rsync		
				- GPL 3
				
			Unison
				- Open Source by Benjamin C. Pierce 
				- Mac, Windows, Linux
				http://www.stanford.edu/~pgbovine/unisonguide.htm
					- batch option
					- ssh
				What's in Unison
				File Synchronization with Unison
					- two-way rsync with a bit of revision control mixed in
					- uses the rsync algorithm to keep network traffic down and should be tunneled through SSH over untrusted networks
					- Unison is programmed in OCaml
				- Own tests
					- Unison will not sync a file currently being written to
					
			Synkron
				- written in C++ and uses the Qt4 libraries
				- GPL v2
				- multiplatform 
				- synchronising multiple folders at the same time
				- can sync subfolders
				- file exclusions maintained in a blacklist; filters
				- scheduler
				- restore files overwritten during sync
				
			DirSync Pro 
				- GPL3
				- Java (JRE 1.6.0 and higher.)
				- Bi-directional (Two way) and mono-directional (One way) synchronization mode
				- Option to synchonise subdirectories recursively
				- Synchronizes files/folders any file system
				- handling time-stamps
				- Schedule Engine
				- gui and create a command line and save it to a batch file
				- logging/reporting facilities
				- no encryption
				
		Open source online tools
			SparkleShare 
				- opensource alternative to Dropbox
				https://github.com/hbons/SparkleShare
				http://sparkleshare.org/
					- Sa collaboration and sharing tool 
					- linux, mac, android
					- Documents synchronised to all peers when changes are made
					- notifications when someone has made a change
					- have as many projects as you'd like
					- use as much space as you'd like
					- run on as many hosts as you'd like
					- uses the GIT system as its backbone
					- store files on own Git server (SSH), Gitub, Gitorious
				http://www.makeuseof.com/tag/sparkleshare-great-open-source-alternative-dropbox-linux-mac/
				
			dvcs-autosync
				http://gitorious.org/dvcs-autosync
				explanatory article: http://www.mayrhofer.eu.org/dvcs-autosync
					- open source replacement for Dropbox/Wuala/Box.net/etc
					- based on distributed version control systems (DVCS)
					- Git is is being tested most thoroughly as the backend storage, but other DVCS such as Mercurial are also supported
					- A single Python script monitors the configured directory for live changes, commits these changes to the DVCS (such as git) and synchronizes with other instances using XMPP messages
					- linux only (since it relies on inotify)
					
			Syncany
				- open-source dropbox alternative
				- encrypts the files locally
				- plug-in based storage system supporting FTP, Amazon S3, Google Storage, Rackspace Cloud Files, WebDAV, Picassa, Box.net
				- Currently immature (alpha not yet released -- Windows & MAC versions are even further behind)
		
		
		Online services
			Windows only
				symform
				http://www.symform.com/resilient-storage-architecture.aspx
					Windows only
				LiveMesh 
					http://www.codeproject.com/Articles/37200/Cloud-Based-Source-Control-using-Live-Mesh-and-Git
			
			SpiderOak
				Free with limited functionality by Spideroak
				Mac, iPhone, iPad, Windows, Linux, Online, Android
				- zero-knowledge privacy approach 
				- SpiderOak operates its own hardware and data centers without outsourcing
				- backup, sync, share
				- SpiderOak keeps historical versions of every file
				- 2048 bit RSA and 256 bit AES
				- \$10/mo/100 GB
				
			Dropbox
				Free with limited functionality 
				Mac, iPhone, iPad, Windows, Linux, Online, Android, Blackberry
				- Dropbox transfers just the parts of a file that change
				- Very common
				- Manually set bandwidth limits
				- Sharing
				- \$10 / month/50GB; \$20 / month/100GB; 
				- Secure Sockets Layer (SSL) and AES-256 bit encryption
				
			Ubuntu One
				Free with limited functionality by Canonical Ltd. 
				Windows, Linux, Online, Android, iPhone
				\$3/mo/20GB
				- Ubuntu One's data storage is simply a set of publically accessible CouchDBs.
				
			SugarSync
				Free with limited functionality by SugarSync 
				Mac, iPhone, iPad, Windows, Win Mobile, Online, Android, S60, Blackberry
				- backup any folder
				- versioning
				-real-time upload of changes
				- no file size limits
				- business plans
				
			Wuala
				Free by LaCie 
				Mac, iPhone, iPad, Windows, Linux, Online, Android
				- encrypts the data on your computer before it is uploaded
				- sync across multiple computers
				- share
				- Business and personal licenses
					- Business: 100 GB for 5 users - Eur 279 /year
					
			CrashPlan
				Free with limited functionality by Code42 
				Mac, Windows, Linux, Online
				- plans per computer or per GB
				
			Box.net
				Free with limited functionality by Box.net
				iPhone, iPad, Windows, Online, Android, Blackberry
				- personal, business and enterprise pricing
					- business: \$15/user/month - 1TB; 2GB file limit
					- Enterprise: \$? unlimited storage; 2 GB file limit
				- file versioning
				- comments and discussions
				- tasks
				- full text search
			Syncplicity
				Free with limited functionality by Syncplicity
				Mac, iPhone, Windows, Online
				- personal edition: \$15/mo/50GB
				- Business edition: \$45/mo/unlimited storage
				- no file size limit
			TeamDrive
				Free by TeamDrive Systems GmbH 
				Mac, Windows, Linux, Online
				- store on their TeamDrive cloud, your own server, or WebDAV servers
				
			ZumoDrive
				Free with limited functionality by Zecter Inc.
				Mac, iPhone, iPad, Windows, Linux, Online, Android
				- 10 GB \$2.99 / month; 25 GB \$6.99 / month; 50 GB \$9.99 / month; 100 GB \$19.99 / month ; 200 GB \$37.99 / month ; 500 GB 	\$79.99 / month
				
			Tonido
				Free by CodeLathe 
				Mac, iPhone, iPad, Windows, Win Phone 7, Linux, Online, Android, Blackberry
				- personal cloud software
				- make files and media in that computer available anywhere 
				- remotely access or share your music, photos, calendar, files, and more from any computer or mobile phone
				- TonidoPlug
					- Connect your TonidoPlug to your router, then connect any external USB hard drive to your plug to access your files, music and media stored in that hard drive from anywhere via any browser.
			Minus
				Free by Minus Inc 
				Mac, iPhone, iPad, Windows, Win Phone 7, Linux, Online, Android, Android Tablet, Blackberry  
				- drag and drop
				
			Duplicati
				Open Source by Kenneth Skovhede, mortenmie, et al 
				Windows, Linux 
				- c#
				LGPL
				- storing the initial copy and then differentials to go from the initial version to the current
				- works with a number of different backends, eg. FTP, WEBDAV and S3, Rackspace Cloud File
				- built-in AES-256 encryption and backups can be signed using GNU Privacy Guard
				- built-in scheduler
				command-line client
				
			Acid Rain
				Open Source
				Mac, Windows, Linux, Online 
				- uses Mercurial 
					- run own server or use a Mercurial hosting service
				- Acid Rain Server is a Linux distribution based on Open Suse 11.3
			Sharebox
				- immature (in development)
				- a filesystem that will synchronize arbitrary data between several machines
				- developed in c
				- storage through git-annex
			lipsync
				- a lightweight commandline service that securely syncronizes your data 
				- linux only (server and client)
				- openSSH
				- rSync-based
				
		Personal Cloud servers		
			Pogoplug
				- £40
				- S/w for PC, Mac, Linux, iPhone, iPad, Android
			Iomega
				ix2-200 and ix4-200d network storage devices

	(Distributed) File Systems
		General Resources
			Cloud-based synchronization of distributed file system hierarchies
		Systems
			HekaFS
			http://git.fedorahosted.org/git/?p=CloudFS.git 
			Coda
			GlusterFS
			RFS
				RFSa network file system for mobile devices and the cloud [102]
		WebDAV 
		FUSE		

	Cloud storage services
		general resources
			Cloud storage survey (unpub.)
			An automated approach to cloud storage service selection
			MetaStoragecamera-ready
			Cloud Storage: Adoption, Practice and Deployment
				- Best practices
					* Chosing a provider: redundancy, fail-over, versioning, data back-up, mgmt console, pricing
					* local storage as well as cloud?
		Providers
			Amazon S3
			Windows Azure
			ATandT Synaptic Storage
			Rackspace CloudFile
			Peer1 CloudOne
			Nirvanix
			Mezeo
			Google Storage
			traditional FTP
			WebDav server
			Pachube
				RESTful
				Environment, Datastream and Datapoint model
				Push and Pull capabilities with ``live'' and ``frozen'' status
				Supports HTTPS/SSL
				Authentication is handled using API keys.


	Database Replication
		Unstructured
			CouchDB
			MongoDB

	Weak connections
		weaklyconnectedusers
		Peer-to-peer Data Replication Meets Delay Tolerant Networking
	Deduplication
		Building a high-performance deduplication system
	
	Cloud data security
		Challenges in Secure Sensor-Cloud Computing
		http://gigaom.com/cloud/the-cloud-meets-the-law-where-wikileaks-went-wrong/
		Structured Encryption and Controlled Disclosure
		Computing Arbitrary Functions of Encrypted Data
		Every Cloud has An Encrypted Lining: The Effectiveness of Cryptography in Cloud Computing	
		Fully homomorphic encryption using ideal lattices
		Distributing data for secure database services
		Addressing cloud computing security issues
		Silverline toward data confidentiality in storage-intensive cloud applications
		[92] Energy-efficient incremental integrity for securing storage in mobile cloud computing
		Lightweight and Compromise Resilient Storage Outsourcing with Distributed Secure Accessibility in Mobile Cloud Computing
		[93] Authentication in the clouds: a framework and its application to mobile users				