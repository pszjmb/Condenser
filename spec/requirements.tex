\section{Requirements}
	This section defines functional and non-functional requirements for the Condenser tool. It begins with an enumeration of usage scenarios that Condenser could be involved in. These are used to help determine the following requirements.

\subsection{Actors}
\begin{center}
	\begin{figure}[htbp]
		\includegraphics[scale=.5]{images/ActorRelationships.png}
		\caption{Actor relationships.\label{ActorRelationships}}
	\end{figure}
\end{center}	
The actors shown in figure~\ref{ActorRelationships} are involved in the following usage scenarios and cases. They include:
\begin{itemize}
	\item Horizon project staff (HPS) -- These actors are involved in research projects as investigators.
	\item Horizon Partner (HP) -- These actors are involved in research projects as investigators. 
	\item Active Ingredient staff (AIS) -- These actors are a type of Horizon Partner. 
	\item Project Participant (PP) -- These actors are involved in research projects as participants. 
	\item Member of the public (MP) -- These actors are a type of Project Participant recruited from the general public.
	\item Energy project participant (EPP) -- These actors are a type of Project Participant recruited in particular for energy projects.
\end{itemize}

\subsection{Usage Scenarios}
			\subsubsection{Relate Project: Condenser Installation}
			\textbf{Participating Actors:}  Anne:AIS , Harry:HPS \\
\textbf{Event Flow:}
	\begin{enumerate}
\item  Anne prepares to run an energy and climate data collection exercise in the Brazilian rain forest using open mobile sensing kits (OMSK).
\item  Harry installs Condenser on the local server component of the OMSKs.
\item  Harry configures a Cloud-server to accept data uploads from each OMSK.
\item  Harry uses the Condenser RESTful interfaces to configure them to log the data from their local stores to the cloud when connectivity is available. He also configures them to run in low-powered mode by setting their reconnection attempts to a low amount and to only send data in aggregated bursts every hour.
	\end{enumerate}
	\line(1,0){350}
			\subsubsection{Relate Project: Oasis of Connectivity}
			\textbf{Overview:} Condenser's behaviour in long period without connectivity \\
			\textbf{Participating Actors:}  Anne:AIS, Miguel:MP \\
\textbf{Event Flow:}
	\begin{enumerate}
\item Anne conducts a public art activity in the Brazilian rain forest.
\item Anne gives Miguel an OMSK.
\item Miguel activates the OMSK as part of the activity.
\item On start up the OMSK activates Condenser and begins to capture temperature, humidity, C02 and energy information.
\item No network connectivity is available to Condenser so it evaluates the expected data load and determines that no action needs to occur until the next connection attempt. [Note that this would be much more complicated if Condenser determined that the current data capture rates exceeded the storage capacity ]
\item Miguel returns the OMSK to Anne, and she returns to her office with it.
\item \textbf{Condenser} successfully attempts to connect to the network after its timeout. 
\item \textbf{Condenser} transfers the data from the OMSK to the Cloud-store.
\item Anne looks at the day's data using the web-based visualisation tools. These have immediate access to the Cloud-data.
	\end{enumerate}
	\line(1,0){350}	
	\subsubsection{Energy Project: Trickle of Connectivity}
			\textbf{Overview:} Condenser's behaviour during intermittent connectivity \\
			\textbf{Participating Actors:}  Harry:HPS, Ernie:EPP \\
\textbf{Event Flow:}
	\begin{enumerate}
\item Harry is conducting an energy study.
\item Harry sets up a Shiva Plug computer with an energy monitor.
\item Harry installs Condenser on the Shiva Plug.
\item Harry configures Condenser to attempt to reconnect whenever possible and to send discrete data points to a rack-mounted server using particular authentication details. 
\item  Harry installs the energy monitor in Ernie's home and configures it to use Ernie's home router.
\item \textbf{Condenser} attempts to transmit data as the energy monitor receives them, but connectivity is intermittent owing to Ernie's unstable Internet connection.
	\end{enumerate}
	\line(1,0){350}
\subsection{Use Cases}
	\subsubsection{Installation}		 
The following use cases pertain to installation of Condenser. Figure~\ref{InstallationUsages} shows a diagram depicting the relationships between the installation use cases.
\begin{center}
	\begin{figure}[htbp]
		\includegraphics[scale=.4]{images/InstallationUse.png}
		\caption{Use cases defining Condenser installation.\label{InstallationUsages}}
	\end{figure}
\end{center}	
\textbf{Use Cases:}\\

		\textbf{Install local server and DB}\\	 
		\textbf{Participating Actors:} Horizon project staff (HPS)  \\
		\textbf{Event Flow:}
		\begin{enumerate}
\item HPS downloads (or compiles from source) the appropriate Condenser binaries from a well-known repository.
\item HPS sets up the Condenser file directory structure on the appliance.
\item HPS sets up Condenser's metadata (such as provenance information). See the setup local metadata use case.
\item HPS downloads and installs the database from a well-known repository (see the Setup DB use case).
\item HPS installs the Condenser service and configures it to cache data in the database.
\item HPS sets Condenser's connection settings (including any public key and password details).
\item HPS sets Condenser's logging settings (see the Setup Logging use case).
	    \end{enumerate}
		\textbf{Entry Conditions:}
		\begin{itemize}
\item The given appliance meets the minimum hardware and operating system requirements.
\item The given appliance is in a clean state (ie. there are no previous installation files/settings that can interfere with this installation).
\item HPS has administrative control (and passwords) of the given appliance.
\item HPS has Internet access.
	    \end{itemize}
		\textbf{Exit Conditions:} Condenser is installed on the given appliance and ready to be externally synchronized.\\
		\textbf{Quality Requirements:}
		\begin{itemize}
\item Condenser is able to connect to the Internet and local sensor network.
\item Data pushed to Condenser is cached appropriately.
\item Condenser is able to push data upstream securely.
		\end{itemize}
		\line(1,0){350}		
		
		\textbf{Set synchronization settings}\\ 
		\textbf{Participating Actors:} Horizon project staff (HPS) \\
		\textbf{Event Flow:}
		\begin{enumerate}
\item HPS accesses the synchronization settings web interface for the given device.
\item HPS sets connection settings to the external (possibly cloud) data storage system (see Set External Storage use case).
\item HPS sets synchronization rules (such as time-based or event-based synchronization). 
\item HPS sets local data cache rules (how to handle limited data storage).
	    \end{enumerate}
		\textbf{Entry Conditions:}
		\begin{itemize}
\item Condenser service is running.
\item External (possibly cloud-based) storage has been setup and may be connected to remotely.
\item HPS has Internet access.
		\end{itemize}		 
		\textbf{Exit Conditions:}
		\begin{itemize}
\item Synchronization between Condenser's local cache and external data storage is up and running.
\item Condenser's synchronization settings are viewable and update-able by an administrator with suitable privileges.
		\end{itemize}			
		\textbf{Quality Requirements:}
		\begin{itemize}
\item Connection settings (such as passwords/keys) to external storage are suitably encrypted.
		\end{itemize}		
		\line(1,0){350}		

		\textbf{Set external storage connection settings}\\ 
		\textbf{Participating Actors:}  Horizon project staff (HPS)\\
		\textbf{Event Flow:}
		\begin{enumerate}
\item HPS accesses the external storage connection settings web interface for the given device.
\item HPS adds one or more connections to the connections store.
\item HPS successfully tests the connection(s).
	    \end{enumerate}
		\textbf{Entry Conditions:} Includes Set Synchronization use case entry conditions.
		\textbf{Exit Conditions:} Condenser is shown to be able to connect and transmit data to external storage.\\
		\line(1,0){350}		
		 
		\textbf{Setup local database} \\		 
		\textbf{Participating Actors:} Horizon project staff (HPS)  \\
		\textbf{Event Flow:}
		\begin{enumerate}
\item HPS downloads the database from a well-known repository.
\item HPS installs the database.
\item HPS sets up the database structure appropriate for the given project or application.
	    \end{enumerate}
		\textbf{Entry Conditions:} No other database is running on the given port.\\
		\textbf{Exit Conditions:} The local database is setup and teted.\\
		\line(1,0){350}
						 
		\textbf{Setup logging} \\	 
		\textbf{Participating Actors:} Horizon project staff (HPS)  \\
		\textbf{Event Flow:}
		\begin{enumerate}
\item HPS accesses the logging settings web interface for the given device. 
\item HPS configures the logging settings.
\item HPS tests to make sure that activity is being logged appropriately.
	    \end{enumerate}
		\textbf{Entry Conditions:} Empty log settings and no log files present.
		\textbf{Exit Conditions:} Condenser is setup to log activity to a given degree of verbosity.
		\textbf{Quality Requirements:}
		\begin{itemize}
\item Different levels of logging are available.
\item Options are avaialable to set the disk size for log storage.
\item Options are avaialble for log cleanup rules.
		\end{itemize}
		\line(1,0){350}
						 
		Setup server metadata \\
		\textbf{Participating Actors:} Horizon project staff (HPS)  \\
		\textbf{Event Flow:}
		\begin{enumerate}
\item HPS accesses the metadata settings web interface for the given device. 
\item HPS configures Condenser's provenance settings.
\item HPS adds any special metadata for the given project or application.
	    \end{enumerate}
		\textbf{Entry Conditions:} Empty metadata settings.\\
		\textbf{Exit Conditions:} Condenser's metadata is setup.\\
		\line(1,0){350}
				 
		Register nodes	\\	 
		\textbf{Participating Actors:} Horizon project staff (HPS)  \\
		\textbf{Event Flow:}
		\begin{enumerate}
\item HPS accesses the node registry web interface for the given device. 
\item HPS registers one or more nodes to capture data for.
\item For each node HPS records data provenance information (such as node type -- ie. type of sensor) and other metadata pertaining to the given node.
\item HPS ensures that each node can transmit data to Condenser and that the given data are stored locally.
	    \end{enumerate}
		\textbf{Entry Conditions:} No nodes are registered with Condenser.\\
		\textbf{Exit Conditions:} one or more nodes is registered with Condenser.\\
		\textbf{Quality Requirements:} It must be possible to quickly register a large number of nodes.\\
		\line(1,0){350}		
	\subsubsection{Local data Capture}		 
		Capture Data\\	 
		\textbf{Participating Actors:}  ... \\
		\textbf{Event Flow:}
		\begin{enumerate}
\item  ...
	    \end{enumerate}
		\textbf{Entry Conditions:}\\
		\textbf{Exit Conditions:}\\
		\textbf{Quality Requirements:}\\
		\line(1,0){350}	
		
		Capture Metadata\\	 
		\textbf{Participating Actors:}  ... \\
		\textbf{Event Flow:}
		\begin{enumerate}
\item  ...
	    \end{enumerate}
		\textbf{Entry Conditions:}\\
		\textbf{Exit Conditions:}\\
		\textbf{Quality Requirements:}\\
		\line(1,0){350}	
		
		Capture Logging Information\\	 
		\textbf{Participating Actors:}  ... \\
		\textbf{Event Flow:}
		\begin{enumerate}
\item  ...
	    \end{enumerate}
		\textbf{Entry Conditions:}\\
		\textbf{Exit Conditions:}\\
		\textbf{Quality Requirements:}\\
		\line(1,0){350}						
	\subsubsection{Data Transfer}		
The following use cases pertain to how Condenser transfers data to the cloud. Figure~\ref{DataTransferUse} shows a diagram depicting the relationships between the data transfer use cases.
\begin{center}
	\begin{figure}[htbp]
		\includegraphics[scale=.5]{images/DataTransferUse.png}
		\caption{Use cases defining Condenser data transfer.\label{DataTransferUse}}
	\end{figure}
\end{center}	
 
		Standard Data Transfer\\	 
		\textbf{Participating Actors:}  ... \\
		\textbf{Event Flow:}
		\begin{enumerate}
\item  ...
	    \end{enumerate}
		\textbf{Entry Conditions:}\\
		\textbf{Exit Conditions:}\\
		\textbf{Quality Requirements:}\\
		\line(1,0){350}			
			 
		Metadata Transfer\\	 
		\textbf{Participating Actors:}  ... \\
		\textbf{Event Flow:}
		\begin{enumerate}
\item  ...
	    \end{enumerate}
		\textbf{Entry Conditions:}\\
		\textbf{Exit Conditions:}\\
		\textbf{Quality Requirements:}\\
		\line(1,0){350}		
			 
		Handling limited data storage concerns \\	 
		\textbf{Participating Actors:}  ... \\
		\textbf{Event Flow:}
		\begin{enumerate}
\item  ...
	    \end{enumerate}
		\textbf{Entry Conditions:}\\
		\textbf{Exit Conditions:}\\
		\textbf{Quality Requirements:}\\
		\line(1,0){350}		
	\subsubsection{Local Service Review}	
The following use cases pertain to the review of service settings and logs of Condenser. Figure~\ref{LocalServiceReviewUse} shows a diagram depicting the relationships between the Local service review use cases.
\begin{center}
	\begin{figure}[htbp]
		%\includegraphics[scale=.5]{images/LocalServiceReviewUse.png}
		\caption{Use cases defining Condenser service review.\label{LocalServiceReviewUse}}
	\end{figure}
\end{center}	
\textbf{Use Cases:}\\
	 
	\textbf{Metadata Review} \\	 
	\textbf{Participating Actors:} Horizon Project Staff(HPS) and/or Horizon Partner (HP)\\
	\textbf{Event Flow:}
	\begin{enumerate}
\item HPS or HP accesses the metadata settings web interface for the given device. 
\item HPS or HP re-configures Condenser's provenance settings.
\item HPS or HP adds any special metadata for the given project or application.
\item Condenser's new settings are saved as a new version. Old versions are archived for future inspection.
    \end{enumerate}
	\textbf{Entry Conditions:} Condenser is running and can be connected to through the Internet.\\
	\textbf{Exit Conditions:} Condenser has new provenance settings.\\
	\textbf{Quality Requirements:} Metadata is maintained under version control.\\
	\line(1,0){350}		
			 
	\textbf{Log Review} \\	 
	\textbf{Participating Actors:} Horizon Project Staff(HPS) and/or Horizon Partner (HP) \\
	\textbf{Event Flow:}
	\begin{enumerate}
\item HPS accesses the logging settings web interface for the given device. 
\item HPS re-configures the logging settings.
\item HPS tests to make sure that activity is being logged appropriately.
    \end{enumerate}
	\textbf{Entry Conditions:} Condenser is running and can be connected to through the Internet.\\
	\textbf{Exit Conditions:} Condenser's logging settings are updated.\\
	\line(1,0){350}			
			 
	\textbf{Update configuration settings} \\	 
	\textbf{Participating Actors:} Horizon Project Staff(HPS) and/or Horizon Partner (HP) \\
	\textbf{Event Flow:}
	\begin{enumerate}
\item HPS accesses the configuration settings web interface for the given device. 
\item HPS re-configures the settings.
    \end{enumerate}
	\textbf{Entry Conditions:} Condenser is running and can be connected to through the Internet.\\
	\textbf{Exit Conditions:} Condenser's settings are updated.\\
	\line(1,0){350}				
	\subsubsection{Decommissioning}		 
The following use cases pertain to the decomissioning of Condenser components. Figure~\ref{DecomissioningUse} shows a diagram depicting the relationships between the decomissioning use cases.
\begin{center}
	\begin{figure}[htbp]
		%\includegraphics[scale=.5]{images/DecomissioningUse.png}
		\caption{Use cases defining Condenser decomissioning.\label{DecomissioningUse}}
	\end{figure}
\end{center}	
\textbf{Use Cases:}\\

	\textbf{Removing a node} \\	 
	\textbf{Participating Actors:} Horizon Project Staff(HPS) and/or Horizon Partner (HP) \\
	\textbf{Event Flow:}
	\begin{enumerate}
\item HPS or HP accesses the node registry web interface for the given device. 
\item HPS or HP un-subscribes to a node.
\item Condenser ceases to receive data from the node.
	    \end{enumerate}
		\textbf{Entry Conditions:} A node is registered with Condenser.\\
		\textbf{Exit Conditions:} The node is no longer registered with Condenser.\\
		\line(1,0){350}

	\textbf{Removing all nodes} \\	 
	\textbf{Participating Actors:}  Horizon Project Staff(HPS) and/or Horizon Partner (HP) \\
	\textbf{Event Flow:}
	\begin{enumerate}
\item HPS or HP accesses the node registry web interface for the given device. 
\item HPS or HP un-subscribes all nodes.
    \end{enumerate}
	\textbf{Entry Conditions:} At least one node is registered with Condenser.\\
	\textbf{Exit Conditions:} No nodes are registered with Condenser.\\
	\line(1,0){350}		

	\textbf{Taking Condenser offline} \\	 
	\textbf{Participating Actors:}  Horizon Project Staff(HPS) and/or Horizon Partner (HP) \\
	\textbf{Event Flow:}
	\begin{enumerate}
\item HPS or HP accesses the uninstall web interface for the given device. 
\item After a suitable warning HPS or HP initiate unistall.
\item Condenser attempts to backup all outstanding data, metadata and longs.
\item Condenser removes its local data storage contents and schema.
\item Condenser removes all of its installed software.
    \end{enumerate}
	\textbf{Entry Conditions:} Condenser is running.\\
	\textbf{Exit Conditions:}\\	
	\begin{enumerate}
\item All data cached by Condenser is replicated elsewhere.
\item Condenser is uninstalled leaving no footprint.
    \end{enumerate}
	\line(1,0){350}			
	%\subsubsection{Features and Plug-ins}		 
	Predictive connectivity \\	 
	\textbf{Participating Actors:}  ... \\
	\textbf{Event Flow:}
	\begin{enumerate}
\item  ...
    \end{enumerate}
	\textbf{Entry Conditions:}\\
	\textbf{Exit Conditions:}\\
	\textbf{Quality Requirements:}\\
	\line(1,0){350}		
 
	Filling in missing data \\	 
	\textbf{Participating Actors:}  ... \\
	\textbf{Event Flow:}
	\begin{enumerate}
\item  ...
    \end{enumerate}
	\textbf{Entry Conditions:}\\
	\textbf{Exit Conditions:}\\
	\textbf{Quality Requirements:}\\
	\line(1,0){350}		

	-- These features may be adde to a future version
	\subsection{Functional Requirements}
\begin{itemize}
\item Condenser is a cloud gateway infrastructural component for use in software projects requiring data transmission and intermittent synchronization with external environments. 

\item Functionality of Condenser is configurable through a RESTful interface. Clients can use HTTP calls to configure where and how often data is stored. Condenser can be used with cloud-based storage or more traditional server storage.

\item Condenser is capable of handling disconnected operation by ensuring data are cached locally until network connection is resumed. Condenser evaluates expected data needs while offline and can be configured to clean, aggregate or delete data to handle limited local drive space.

\item Condenser uses a web-based interface to set certain configuration settings. 

\item Condenser will automatically provide standard metadata.

\item Condenser will log its own performance to an external repository (with configurable repository and verbosity).

\item Condenser should run as a background service that turns on automatically at startup
\end{itemize}
	\subsection{Non-functional Requirements}
		\subsubsection{Reliability and Security}
Condenser should continue to work reliably during periods of network disconnect. Error conditions that arise during periods of disconnect should be logged in a similar fashion to sensor data and transmitted externally (to a configurable store) upon resumption of network connection. Configurable options should be made available to allow for secure data transmission and storage. Data store connection settings must be encrypted.		
		\subsubsection{Test Data}
Even though Condenser will be data-type neutral, it will be tested using temperature, humidity, energy and C02 sensors data. Data sets will include: time-series data, discreet values and aggregate information. Condenser will be tested with up to 2GB of data.		
		\subsubsection{Performance}
Condenser'	s external storage performance during times of reliable network connection will be configurable is order for an administrator to choose the amount of bandwidth will be taken up transferring data offsite. Condenser should support the offsite data transfer of up to 4095 sources as well as one more for its own logging information.

When dealing with external storage, there is a bounded relationship between cost, capacity and latency. These factors need to be balanced for any Condenser solution. 	
		\subsubsection{Supportability}
Condenser tests, code, installation, administration and usage will be well documented.	
		\subsubsection{Implementation}
The local logging Condenser component will be operational on a Plug computer. Off the shelf technology may be (and indeed is encouraged to be) used to get a working version of Condenser built as soon as possible. New developments should be test-driven with an emphasis on documentation. 		
		\subsubsection{Interface}
Condenser should support a RESTful interface for its activation, configuration and data transmission. Condenser should be compatible with other Relate, HomeWork and Energy project components.

Condenser should be able to store data externally using a common solutions such as	Amazon S3, Windows SQL Azure and FTP.
		\subsubsection{Operation}
Condenser will be managed by Horizon staff or project clients.		
		\subsubsection{Packaging}
Condenser will be installed initially by Horizon project staff, with a view that client system administrators will be supported in the future. Condenser will initially be rolled-out for beta-testing in February, 2011.  		
		\subsubsection{Legal}
Condenser will be licensed under The GNU Affero General Public License (AGPL 3). No liability will be assumed by Condenser's developers for losses incurred through its use since it is experimental and research driven software. Only Free (libre) software will be used for the development of Condenser.		